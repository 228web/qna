\documentclass[a4paper,12pt]{article}
\usepackage[utf8]{inputenc}
\usepackage[T1]{fontenc}
\usepackage{graphicx}
\usepackage{amssymb}
\usepackage[french]{babel}

\def\N{\mathbb{N}}

\title{Questions adaptatives}
\author{Jean-Bastien Grill \and Jill-Jênn Vie}

\begin{document}
\maketitle

On considère une population d'individus numérotés $m = 1, \ldots, n$, chacun ayant une capacité $k_m \in \N$ parmi un ensemble $C$.

$\alpha_{ij}, \beta_{ij}$ connus.

$p_{ij}$ inconnus $\sim B(\alpha_{ij}, \beta_{ij})$ indépendants.

$k_m$ iid $\sim$ un a priori $\pi$

\paragraph{Mécanique.} Séquence de questions posées aux individus.

$q_{mt} \in \N, t = 1, \ldots, T$

Il répond $r_{mt} \sim Ber(p_{q_{mt}, k_m})$.

\paragraph{Objectif.} Estimation efficace des $k_m$ ?

\end{document}
